%%%%%%%%%%%%%%%%%%%%%%%%%%%%%%%%%%%%%%%%%
% Jiajun Yao's resume
%
% Original author:
% Rensselaer Polytechnic Institute (http://www.rpi.edu/dept/arc/training/latex/resumes/)
%
% Important note:
% This template requires the resume.cls file to be in the same directory as the
% .tex file. The resume.cls file provides the resume style used for structuring the
% document.
%
%%%%%%%%%%%%%%%%%%%%%%%%%%%%%%%%%%%%%%%%%

%----------------------------------------------------------------------------------------
%	PACKAGES AND OTHER DOCUMENT CONFIGURATIONS
%----------------------------------------------------------------------------------------

%!TEX TS-program = xelatex
%!TEX encoding = UTF-8 Unicode

\documentclass[mm, 7pt]{resume} % Use the resume.cls style, the font size can be changed to 11pt or 12pt here

%----XeTeX start
% FONTS
\usepackage{xunicode}
\usepackage{xltxtra}
\defaultfontfeatures{Mapping=tex-text} % converts LaTeX specials (``quotes'' --- dashes etc.) to unicode
% if want to use specific weight as default, add UprightFont={* Light} option
% for detail, see http://tex.stackexchange.com/q/62603
\setromanfont[Ligatures={Common}, Numbers={OldStyle}]{Palatino} %candidates: Palatino, Baskerville, Latin Modern Math,
\setmonofont[Scale=0.8]{Monaco} 
\setsansfont[Scale=0.9]{Optima Regular}
%----XeTeX end

%\usepackage{helvet} % Default font is the helvetica postscript font
%\usepackage{newcent} % To change the default font to the new century schoolbook postscript font uncomment this line and comment the one above

\addtolength{\oddsidemargin}{-0.45in}
\addtolength{\textwidth}{0.9in} 
\addtolength{\textheight}{1.0in}
\addtolength{\topmargin}{-0.6in}
\setlength{\parskip}{0.7em}

\begin{document}
%----------------------------------------------------------------------------------------
%	NAME AND ADDRESS SECTION
%----------------------------------------------------------------------------------------

\moveleft.5\hoffset\centerline{\large\bf Jiajun Yao} % Your name at the top
\moveleft.5\hoffset\centerline{1 (412) 951-0068}
\moveleft.5\hoffset\centerline{jeromeyjj@gmail.com}
\moveleft.5\hoffset\centerline{http://blog.jjyao.me/}
\moveleft.5\hoffset\centerline{https://www.linkedin.com/in/jjyao/}
\moveleft\hoffset\vbox{\hrule width 7.4in height 0.8pt} % Horizontal line; adjust line thickness by changing the '1pt'
\vspace{-0.1in}
%----------------------------------------------------------------------------------------

\begin{resume}
%----------------------------------------------------------------------------------------
%	WORK EXPERIENCE SECTION
%----------------------------------------------------------------------------------------

\section{WORK EXPERIENCE}

\textbf{Anyscale Ray Core Team} \hfill August 2021 - Present \\
{\sl Tech Lead Manager}  \hfill San Francisco, CA
\begin{itemize} \itemsep -2pt
\item[-] Senior Ray committer with 600+ commits
\item[-] Rebuilt the Ray Core team, growing it from the group up to over 15 members across multiple continents
\item[-] Led and contributed to various Ray Core improvements across stability, scalability, performance, observability and usability such as improving Ray Core scalability from 4k nodes to 12k nodes, making Ray Core fault tolerant to node and network failures and supporting label-based scheduling
\end{itemize}

\textbf{LinkedIn Distributed Graph Team} \hfill February 2016 - August 2021 \\
{\sl Staff Software Engineer}  \hfill Sunnyvale, CA
\begin{itemize} \itemsep -2pt
\item[-] Work on the next generation graph database LIquid that supports the entire LinkedIn economic graph with millions of QPS and sub-second latency:
\item[ ] https://engineering.linkedin.com/blog/2020/liquid-the-soul-of-a-new-graph-database-part-1
\item[-] Work on the entire stack of the system: storage, query evaluation and frontend
\item[-] Leading a group of two people on the traffic migration project: migrate all client traffic to the frontend API seemlessly
\item[-] Working on the declarative query evaluation project: distributed, dynamic and declarative query evaluation engine using Prolog
\item[-] Led a group of four people on the frontend API project: launched the only external RESTful API used by all clients with ACL, quota and monitoring features
\item[-] Led the third-party graph project: launched the first third-party LIquid instance through collaboration with a cross-org client team
\item[-] Led the imperative query evaluation project: served 100\% LIquid traffic with an imperative query language and an imperative query evaluation engine
\item[-] Designed and developed index parallel optimization: optimize running database indexes with no downtime
\end{itemize}

\textbf{Qiduo Information Technology Co., Ltd.}  \hfill  July 2012 - July 2014 \\
{\sl Software Engineer, Founding member}                          \hfill  Beijing, P.R. China
\begin{itemize} \itemsep -2pt
\item[-] Full stack developer
\item[-] Developed a cloud-based new email checking and notification system that served >100K users
\item[-] Developed an Android email app LightMail that supports POP3 and IMAP
\end{itemize}

%----------------------------------------------------------------------------------------
%	PROJECT EXPERIENCE SECTION
%----------------------------------------------------------------------------------------

%\section{SELECTED PROJECTS}
%
%\textbf{L4 Compiler} \hfill August 2015 - December 2015 \\
%{\sl Team Member (2 developers)} \hfill Carnegie Mellon University
%\begin{itemize} \itemsep -2pt
%\item[-] Designed and implemented a compiler for the programming language L4
%\item[-] Implemented some compiler optimizations: peephole, CSE, function inlining, constant folding, constant propagation, copy propagation and dead code elimination
%\item[-] Implemented a mark-and-sweep garbage collector for L4
%\end{itemize}
%
%\textbf{Kernel} \hfill February 2015 - April 2015 \\
%{\sl Team Member (2 developers)} \hfill Carnegie Mellon University
%\begin{itemize} \itemsep -2pt
%\item[-] A Unix-like kernel that supports paging, preemptive multitasking and system calls
%\item[-] Designed and implemented a scheduler, drivers and synchronization primitives
%\item[-] Learned a lot about synchronization, deadlock, parallelism, process, thread and virtual memory
%\item[-] Got A in 15-410
%\end{itemize}
%
%\textbf{MapReduce Framework}  \hfill  October 2014 - November 2014 \\
%{\sl Team Member (2 developers)}   \hfill  Carnegie Mellon University
%\begin{itemize} \itemsep -2pt
%\item[-] A MapReduce framework that is distributed and fault-tolerant
%\item[-] Investigated the design and implementation of Hadoop and HDFS
%\item[-] Architected and developed a distributed file system like HDFS
%\item[-] Architected and developed a MapReduce facility like Hadoop
%\end{itemize}
%
%\textbf{Congestion Control with BitTorrent} \hfill October 2014 - November 2014 \\
%{\sl Team Member (2 developers)} \hfill Carnegie Mellon University
%\begin{itemize} \itemsep -2pt
%\item[-] A BitTorrent-like file transfer application on top of UDP
%\item[-] Architected and developed the server-side of the application
%\item[-] Implemented congestion control mechanisms: slow start, AIMD and fast retransmission
%\end{itemize}

%----------------------------------------------------------------------------------------
%	RESEARCH EXPERIENCE SECTION
%----------------------------------------------------------------------------------------
 
\section{RESEARCH EXPERIENCE}

\textbf{Nanjing University}       \hfill  May 2011 - May 2012 \\
{\sl Undergraduate Research Assistant}  \hfill  Nanjing, P.R. China
\begin{itemize} \itemsep -2pt
\item[-] Applied data mining related techniques to solve the user cold start problem for e-commerce websites using search keywords
\end{itemize}

%----------------------------------------------------------------------------------------
%	EDUCATION SECTION
%----------------------------------------------------------------------------------------
 
\section{EDUCATION}

\textbf{Carnegie Mellon University}, Pittsburgh, PA \hfill December 2015 \\
School of Computer Science, Master of Computational Data Science \hfill GPA 4.03/4.0
\begin{itemize} \itemsep -2pt
\item[-] Took classes in compiler, operating system, distributed system, network, storage
\item[-] Designed and developed a kernel that supports paging, multitasking and system calls
\item[-] Designed and developed a compiler for the L4 programming language with various optimizations: CSE, function inlining, constant folding, constant propagation, dead code elimination, etc
\end{itemize}

\textbf{Nanjing University}, Nanjing, P.R. China \hfill June 2013 \\
School of Software Engineering, Bachelor of Engineering \hfill GPA 90.2/100

%----------------------------------------------------------------------------------------
%	SKILLS SECTION
%----------------------------------------------------------------------------------------
\section{SKILLS}

Programming Languages:             \hfill  C++, Python, Java, Rust, PHP, JavaScript  \\
Operating System:     \hfill  Linux \\
%Big Data Technology:   \hfill Hadoop, Spark, Kafka, Samza, Pig \\
%Mobile Technology:    \hfill  Android \\
%Web Technology:       \hfill  Netty, Nginx, Twisted \\
%Database Experience:  \hfill  MySQL, SQLite \\

\end{resume}
\end{document}
